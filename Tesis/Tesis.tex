\documentclass[letterpaper,12pt,oneside]{book}
\usepackage[top=1in, left=1.25in, right=1.25in, bottom=1in]{geometry}
\usepackage{bachelorstitlepageUNAM}

\author{Luis Esteban Serrano Bermúdez}
\title{}
\faculty{Facultad de Ingeniería}
\degree{Ingeniero en Computación}
\supervisor{Dr. ... Tutor}
\cityandyear{Ciudad Universitaria, Cd. Mx., 2020}
\logouni{Escudo-UNAM}
\logofac{Escudo-FING}

\usepackage[T1]{fontenc}
\usepackage[utf8]{inputenc}
\usepackage[spanish,es-nodecimaldot,es-tabla]{babel}
\usepackage{graphicx}
\usepackage{tikz}

\graphicspath{{./figs/}}
\usepackage{setspace}

\begin{document}
	\frontmatter
	\maketitle
	\chapter*{}

	\begin{flushright}%
	  \emph{Dedicatoria ...}
	  \thispagestyle{empty}
	\end{flushright}

	\chapter{Agradecimientos}
	\spacing{1.5}

	\chapter{Resumen}

	\tableofcontents
	\listoffigures

	\chapter{Prólogo}
	    Este proyecto se ha desarrollado con la finalidad de añadir una mejora a los medidores de energía previamente realizados en el Instituto de Ingeniería de la UNAM. La finalidad de esta mejora es agregar una función de lectura de datos de registros del medidor de energía ADE7880 para poder analizar una gran cantidad de datos en poco tiempo.

	\mainmatter

	\chapter{Introducción}
		
		\section{Computadora}
			Una computadora esta conformada por hardware y software. La parte de Hardware consta de 4 componentes: el procesador, que funciona como el cerebro; la unidad de entrada, por la cual los programas y los datos son ingresados; la unidad de salida por donde son presentados los los resultados y la memoria que es en donde se almacena el software y los datos.

			\subsection{Procesador}
				El procesador, tambien llamada como la Unidad Central de Procesamiento se puede clasificar en tres partes:
				
				\begin{itemize}
					\item \textbf{Registros}: Es una locación de almacenamiento dentro de la CPU en la cual se mantienem los datos y las direcciones de memoria durante la ejecución de una instrucción. Accesar a los registros de datos es más rápido que acceder a los datos en la memoria externa. Estos registros varían dependiendo del modelo del procesador.

					\item \textbf{Unidad Lógica Aritmética}: Es la calculadora numérica y evaluadora lógica de operaciones. Aquí se reciben los datos provenientes de la memoria principal o de los registros, realiza una operación lógica y si es necesario reescribe el resultado de vuelta al registro o memoria.

					\item \textbf{Unidad de Control}: Contiene las instrucciones lógicas del hardware, esta se encarga de decodificar y monitorear la ejecución de las instrucciones. Tambien funciona como árbitro de varios de los servicios del CPU, los cuiales se encuentran sincronizados por un reloj de sistema.
				\end{itemize}
	
	\section{Microprocesador}
		El procesador en una computadora, esta comprendido de varios circuitos integrados, mientras que un microprocesador es un procesador empaquetado en un único circuito integrado. Una microcomputadora usa un microprocesador como su CPU.

		Los microprocesadores vienen en diferentes presentaciones, 4-Bits, 8-Bits, 16-bits, 32-Bits e incluso en 64-Bits aunque estos últimos no son demasiado comunes como los anteriores. El número de bits corresponde al número de digitos binarios que el microprocesador puede manipular en las operaciones.

		El acceso de la memoria principal toma mucho más tiempo que el tiempo de reloj disponible por el CPU, por ello es que los microprocesadores de 32 y 64 Bits poseen una textit{memoria caché} de alta velocidad

	\section{Firmware}
		\subsection{Interfaces}
			\subsubsection{SPI (Serial Peripheral Interface)}

			\subsubsection{Interfaz HSDC (High Speed Data Capture)}
				Debido a que para este proyecto es necesario el recopilar una gran cantidad de muestras a alta velocidad la documentación del Circuito Integrado ADE7880 recomienda, para la lectura de estos registros en específico, utilizar una interfaz propia de Analog Devices para este circuito llamada High Speed Data Capture (Captura de Datos a Alta Velocidad) por lo que es necesario configurar el microprocesador de manera que se pueda usar la comunicación I2C como interfaz serial principal y la comunicación HSDC como secundaria usando el canal SPI del microcontrolador como esclavo, que recibirá toda la información de los registros de voltaje y corriente enviados por esta interfaz.

				Para configurar el puerto HSDC se debe escribir primero al registro HSDC\_CFG [0xE706] a traves del puerto I2C la configuración con la que se van a estar enviando los datos a traves del puerto HSDC. Ya que se configura el puerto se habilita colocando el bit 6 (HSDCEN) en el registro CONFIG [0xE618] a 1, con esto se activa la comunicación	.

				La configuracion que se ha decidido usar es tener el relog a 8[MHz] con la transmision de registros en paquetes de 32-bits, no se agrega una brecha de 7 ciclos de reloj entre transmisiones y unicamente se transmitira el contenido de los registros de voltaje y corriente: IAWV, VAWV, IBWV, VBWV, ICWV, VCWV, e INWV, lo que permite que se realice la comunicación de manera más rápida. El pin de selccion de esclavo (SS ó Chip Select) se mantiene como activo en bajo. %AUN SE ESTA PROBNADO SI FUNCIONARIA CON ACYIVO E BAJO O SE DEBE USAR COMO ACTIVO EN ALTO

			\subsubsection{Interfaz I2C (Inter Integrated Circuits)}


	\chapter{Resultados}
		Como se explicó al principio el desarrollo de este proyecto se pensó como una mejora a los medidores de energía anteriormente fabricados por el Instituto de Ingeniería por lo que este, solo se centró únicamente en la lectura de muestras a alta velocidad por parte del microcontrolador.

		En un principio el sistema medidor de energía estaba pensado para realizar la comunicación entre el microprocesador y el medidor ADE7880 con interfaz SPI sin embargo al momento de realizar las primeras pruebas, los registros correspondientes a los canales de medición de voltaje y corriente, pasado un tiempo los valores que se recuperan de los registros pierden coherencia lo que hace que los resultados sean inutiles para su análisis.

	\chapter{Conclusiones}

	%\bibliographystyle{humannat}
	%\bibliography{references}

	\backmatter%@sglvgdor
\end{document}